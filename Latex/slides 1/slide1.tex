\documentclass{beamer}
\title{Debugging effectiveness summary}
\author{Alessandro Valentini}
\date{06.07.2015}

\newcommand{\itembf}[1]{\item \textbf{#1}}


\begin{document}
\begin{frame}
\maketitle
\end{frame}

\begin{frame}
	\frametitle{Summary}
	\framesubtitle{Intesting points}
	\begin{itemize}
	\item Hypothesis
	\item Definition and planning
	\item Results
	\item Statistical consideration
	\end{itemize}
\end{frame}

\begin{frame}{Hypothesis}
\begin{itemize}
	\item There is no difference in the \textbf{effectiveness} (number of correctly fixed faults) of debugging between autogen e manual test cases
	\item There is no difference in the \textbf{efficiency} (corrected task per time) of debugging between autogen e manual test cases
	\end{itemize}
\end{frame}

\begin{frame}{Definition and planning}
The experiment:
	\begin{itemize} 
		\item compares manual test cases to Randoop and EvoSuite test cases
		\item inspects the impact of obfuscated identifiers
		\item subjects are divided in two groups balaced accordingly experience and ability
		\item two different source code are used (XML-security and JTopas), each group has to debug a specific source using either manual or autogen tests.
	\end{itemize}
\end{frame}

\begin{frame}{Results I}
	\begin{itemize} 
		\itembf{MvR:} Efficiency and effectiveness are improved in particular on experienced developer (Eclipse Debugger), autogen test are easier to understand
		\itembf{MvE:} Effectiveness shows no significant differences, efficiency is increased by manual test cases. Understandability is lower than Randoop test but do not influence debug
		\itembf{MvO:} Obfuscation strongly decrease understandability of manual test cases, however this has no impact on debug
	\end{itemize}
\end{frame}

\begin{frame}{Results II}
	\begin{itemize} 
		\item Autogen test cases are simpler than manual and more useful for unesperienced developers
		\item Experienced developer perform well with both cases and can take advantage from advanced tools
		\item Unexperienced developer use more time to understand test cases decreasing their performance 
		\item In general autogen test cases are useful because can be generated quickly
		\item In general debug should be demanded to experienced developers
	\end{itemize}
\end{frame}

\begin{frame}{Statistical consideration}
	\begin{itemize} 
		\itembf{Confusion Matrix:} express the level of reliability of an algorithm 
		\itembf{Null hypothesis:} No relationship between two events
		\itembf{Likert scale:} Express feelings (level of agreement) respect to an object
		\itembf{Statistically significant: } confidence interval etc to review
		\itembf{Threats to validity:} no reward, order of tasks, few generator (free)
	\end{itemize}
\end{frame}

\end{document}