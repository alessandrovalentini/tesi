\chapter{Debugging effectiveness}
\section{Introduction}
	The experiment investigates the effectiveness of automatically generated test cases with respect to manual test. Autogen are less understandable but simpler, manual test are more complicated and cover even high-level requirement functionalities. \cite{ceccato2015debuggingEffectivenessEfficiency}
	
	\subsection{Definition and planning}
	The test has been divided in three experiments:
	\begin{itemize}
		\itembf{Manual vs. Randoop: } test cases produced by Randoop with the support of Eclipse Debugger
		\itembf{Manual vs. EvoSuite: } test cases auto-generated by EvoSuite
		\itembf{Manual vs. Obfuscated: } investigates whether meaningless identifiers have any impact on debugging, every variable is substituted by x followed by an incremental number.
	\end{itemize}
	Each experimenter investigates effectiveness, efficiency and understandability of test cases.\\
	Eight test cases have been generated, and sometime manually modified, on two different source code. MvR and MvE have only 2 common tests.
	\begin{itemize}
		\itembf{ability:} based on related academic courses and results in training lab
		\itembf{experience:} as BSc students, MSc sudents, PhD/Post-docs, researchers/professors.
		\itembf{object system:} two different source code (JTopas and XML-security) are submitted to developers.
		\itembf{experiment session}
		\itembf{fault to be fixed:} faults have different complexity level
	\end{itemize}
	Subject are divided in two balanced groups alternatively assigned to the same task with or without autogen support. 17 close question have been asked with answers on a Likert scale (strngly agree, agree, uncertain disagree, strongly disagree) in the case of MvR, and 5 closed question plus 2 open questions in the case of MvE and MvO.
	
	\subsection{Hypothesis}
	\begin{itemize}
		\itembf{Effectiveness: } there is no difference in the effectiveness (number of correctly fixed faults) of debugging between autogen e manual test cases
		\itembf{Efficiency: } there is no difference in the efficiency (corrected task per time) of debugging between autogen e manual test cases
	\end{itemize}
	
\section{Results}
	\subsection{Results MvR}
	\subsubsection{Effectiveness}
		Autogen improve effectiveness respect to manually written tests, in particular high experienced subjects take more advantace of autogen test.
		
	\subsubsection{Efficency}
		Efficiency is improved when autogen tests are used, in particular expert developers. The difference in efficiency between experienced ad unexperienced developers is higher than in effectiveness test case.
	
	\subsubsection{Understandability}
	The presence of meaningless identifiers seems not affect the understandability of autogen test cases while manually test cases result harder to understand despite the presence of more meaningful identifiers. This is related to the greater complexity of manual test cases.\\
	According to the post questionnaire low ability subjects could take advantage only of simple test cases.
	
	\subsection{MvE}
	\subsubsection{Effectiveness}
		Differences are statistically significant only in some cases, so the hypothesis that manual test are equally efficient than autogen test cannot be rejected. Difference in experience is less significant, on the contrary the system to debug strongly influence results.
		
	\subsubsection{Efficency}
		Experience did not significantly influence efficiency while manual test cases reduce it. As in effectiveness experiment the system to debug lead to very different results.
		
	\subsubsection{Understandability}
		EvoSuite testcases result simpler but harder to understand than Randoop, however this differences do not influence neighter effectiveness nor efficiency.
		
	\subsection{MvO}
	\subsubsection{Effectiveness}
		The use of obfuscated identifiers in manual test cases do not show any statistical significance and the chance of a Type II error is high.
		
	\subsubsection{Efficency}
		The test does not reach a statistical significance and the probability of a Type II error is high.
		
	\subsubsection{Understandability}
		Understandability result significantly affected by identifiers obfuscation, however a lower understandability does not result in a lack of efficiency or effectiveness.
	
	\section{Conclusions}
	\subsection{Evidence}
	\begin{itemize}
	
		\item Autogen test cases are simpler than manual
		\item They are particularly useful for unexperenced subjects
		\item Experienced subjects performed equally well with both autogen and manual test cases
		\item Less experienced subjects tried to understand the purpose of the analized test
	\end{itemize}
	The performance subjects is distributed along a spectrum that nicely follows their levels of ability and experience.	Autogen tests, in any case, are useful because simpler but faster to generate. Debug purposes should be demanded to experienced developer who are most efficient and effective. The usage of manual test cases respect to autogen test cases has no major impact on senior developer\\
		
	
	\subsection{Threats to validity}
	\begin{itemize}
		\item subject are not rewarded nor evaluated
		\item the order of tasks may influence the difficulty
		\item other generator could be used, but this two are the only freeware
	\end{itemize}